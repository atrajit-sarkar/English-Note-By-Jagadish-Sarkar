\chapter{Sentence}
\section{Definition}

\begin{bengali}
এক  বা   একাধিক  word  পাশাপাশি    বসে  সম্পূর্ণ   অর্থ     বা   মনের ভাব প্রকাশ     করলে     তাকে বলে   sentence. 

যেমন :-
\end{bengali}

\ex{
    \begin{enumerate}
        \item The cow gives us milk.
        \item He is my brother
        \item The sun rises in the east.
    \end{enumerate}
}

\note{📌}{Note:-}{
    \RightBracketBlock{
            \text{Come - \textbengali{এস । }} \\
            \text{go - \textbengali{যাও । }}
            }
            {
               \text{\textbengali{ইহারও  sentence.   তবে  এই  সকল     ক্ষেত্রে    subject     "you" ঊহ্য   থাকে।  }}
            }
}

\imagebox{atrajit-smile.png}{Smile for a reason:}{
    All the original text are written by my Grandpa Jagadish Sarkar and the translations are by me, Atrajit Sarkar.
}

\section{Subject and Predicate}
\begin{bengali}
    Sentence এর যে অংশে যাহাকে উদ্দেশ্য করিয়া বা যাহার সম্বন্ধ্য়ে কিছু বলা হয় তাহাকে \highlight{subject} বলে।
    
    Sentence এর যে অংশে subject সম্বন্ধ্য়ে কিছু বলা হয় তাহাকে বলে \highlight{predicate}.  
\end{bengali}

\begin{table}[h]
  \centering
  \setlength{\tabcolsep}{12pt}     % horizontal padding
  \renewcommand{\arraystretch}{1.4}% vertical padding
  \arrayrulecolor{tableline}

  \begin{tabular}{L{0.35\linewidth} L{0.5\linewidth}}
    \rowcolor{headerbg}
    {\color{headertext}\bfseries Subject} &
    {\color{headertext}\bfseries Predicate} \\
    \midrule
    \rowcolor{subjectcol}
    \textbf{He} &
    plays cricket. \\

    \rowcolor{predicatecol}
    \textbf{The sun} &
    rises in the east. \\

    \rowcolor{subjectcol}
    \textbf{The cow} &
    gives us milk. \\

    \rowcolor{predicatecol}
    \textbf{Ashoka, the emperor of India} &
    loved his subjects. \\
    \bottomrule
  \end{tabular}

  \caption{Examples of Subjects and Predicates}
\end{table}

\section{Forms of Sentences}
\begin{bengali}
    মনের ভাব প্রকাশের বিভিন্ন ভঙ্গি অনুসারে sentence গুলিকে ছয় ভাগে ভাগ করা হইয়াছে । 
    
    যথা   :-
\end{bengali}

\begin{enumerate}
    \item Assertive Sentence
    \item Interogative Sentence
    \item Imprerative Sentence
    \item Optative Sentence
    \item Exclamatory Sentence
    \item Imphatic Sentence
\end{enumerate}

\subsection{Assertive Sentence}
\begin{bengali}
    যে sentence দ্বারা সাধারণ ভাবে কিছু বলা বা বর্ণনা করা হয় তাহাকে \highlight{Assertive sentence} বলে ।

    Assertive Sentence দুই প্রকার :

    \begin{enumerate}
        \item \highlight{Affirmative Sentence} - (হাঁ-বাচক) sentence 
        \ex{
            He gave me a pen.\\
            The earth is round.\\
            Man is mortal.
        }
        \item \highlight{Negative Sentence} (না -বাচক )Sentence 
        \ex{
            He does not take tea.\\
            Ram coul not do the sum.\\
            They are not happy.
        }
    \end{enumerate}

    \imagebox{dadu-note.png}{Assertive Sentence এর গঠন :-}{
        Subject + finite verb + object
    }
\end{bengali}

\subsection{Interogative Sentence}

\begin{bengali}
    যে sentence দ্বারা কোন প্রশ্ন জিজ্ঞাসা করা বুঝায় তাহাকে \highlight{Interogative Sentence} বলে। 

    \note{📝}{গঠন:-}{
        Interogative pronoun বা Adverb $+$ Auxiliary verb $+$ Subject $+$ Finite verb $+$ Object $+$ Interogative sign(?).
     }

     \term{Interogative Pronoun}:- Who, which, what.

     \RightBracketBlock{
        \text{Who did the work?}\\
        \text{Which pen do you want?}\\
        \text{What do you want?}
     }{
        \text{\highlight{Interogative pronoun} \textbengali{সহযোগে গঠিত।}}
     }

     \term{Interogative Adverb}:- Where, when, why, how.

     \RightBracketBlock{
        \text{Where do you go?}\\
        \text{When will you come?}\\
        \text{Why did you do the work?}\\
        \text{How will you solve the problem?}
     }{
        \text{\highlight{Interogative Adverb} \textbengali{সহযোগে গঠিত।}}
     }



     \imagebox{dadu-important.png}{Interogative pronoun "who"}
     {
        দিয়ে যে সমস্ত প্রশ্নবোধক বাক্য রচিত হয় , সে সব ক্ষেত্রে present indifinite এবং past indifinite tense-এ কোন সহাজ্যকারী ক্রিয়া (auxiliary verb) লাগে না। কিন্তু এই দুইটি tense ছাড়া বাকি সমস্ত tense-এ, tense-এর নিজস্ব অর্থ এবং চিহ্ন বজায় রাখার জন্য auxiliary verb লাগে।   
    }\\



    \RightBracketBlock{
        \text{Who goes to market?}\\
        \text{Who eats rice?}\\
        \text{Who play in the field?}\\
        \text{Who wants the book?}
    }{
        \text{present indifinite tense}
    }

    \RightBracketBlock{
        \text{Who told you the story?}\\
        \text{Who took coffee?}\\
        \text{Who invented television?}\\
        \text{Who got the first prize?}\\
        \text{Who stood first in the class?}
    }{
        \text{past indifinite tense}
    }

\end{bengali}

\begin{bengali}
    \begin{enumerate}
        \item Who is singing a song?
        এখানে Continuous tense গঠন করার জন্য auxiliary verb 'is' এসেছে । 
    \end{enumerate}
\end{bengali}
