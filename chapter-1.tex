\chapter{Sentence}
\section{Definition}

\begin{bengali}
এক  বা   একাধিক  word  পাশাপাশি    বসে  সম্পূর্ণ   অর্থ     বা   মনের ভাব প্রকাশ     করলে     তাকে বলে   sentence. 

যেমন :-
\end{bengali}

\ex{
    \begin{enumerate}
        \item The cow gives us milk.
        \item He is my brother
        \item The sun rises in the east.
    \end{enumerate}
}

\note{📌}{Note:-}{
    \RightBracketBlock{
            \text{Come - \textbengali{এস । }} \\
            \text{go - \textbengali{যাও । }}
            }
            {
               \text{\textbengali{ইহারও  sentence.   তবে  এই  সকল     ক্ষেত্রে    subject     "you" ঊহ্য   থাকে।  }}
            }
}

\imagebox{atrajit-smile.png}{Smile for a reason:}{
    All the original text are written by my Grandpa Jagadish Sarkar and the translations are by me, Atrajit Sarkar.
}

\section{Subject and Predicate}
\begin{bengali}
    Sentence এর যে অংশে যাহাকে উদ্দেশ্য করিয়া বা যাহার সম্বন্ধ্য়ে কিছু বলা হয় তাহাকে \highlight{subject} বলে।
    
    Sentence এর যে অংশে subject সম্বন্ধ্য়ে কিছু বলা হয় তাহাকে বলে \highlight{predicate}.  
\end{bengali}

\begin{table}[h]
  \centering
  \setlength{\tabcolsep}{12pt}     % horizontal padding
  \renewcommand{\arraystretch}{1.4}% vertical padding
  \arrayrulecolor{tableline}

  \begin{tabular}{L{0.35\linewidth} L{0.5\linewidth}}
    \rowcolor{headerbg}
    {\color{headertext}\bfseries Subject} &
    {\color{headertext}\bfseries Predicate} \\
    \midrule
    \rowcolor{subjectcol}
    \textbf{He} &
    plays cricket. \\

    \rowcolor{predicatecol}
    \textbf{The sun} &
    rises in the east. \\

    \rowcolor{subjectcol}
    \textbf{The cow} &
    gives us milk. \\

    \rowcolor{predicatecol}
    \textbf{Ashoka, the emperor of India} &
    loved his subjects. \\
    \bottomrule
  \end{tabular}

  \caption{Examples of Subjects and Predicates}
\end{table}

\section{Forms of Sentences}
\begin{bengali}
    মনের ভাব প্রকাশের বিভিন্ন ভঙ্গি অনুসারে sentence গুলিকে ছয় ভাগে ভাগ করা হইয়াছে । 
    
    যথা   :-
\end{bengali}

\begin{enumerate}
    \item Assertive Sentence
    \item Interogative Sentence
    \item Imprerative Sentence
    \item Optative Sentence
    \item Exclamatory Sentence
    \item Imphatic Sentence
\end{enumerate}

\subsection{Assertive Sentence}
\begin{bengali}
    যে sentence দ্বারা সাধারণ ভাবে কিছু বলা বা বর্ণনা করা হয় তাহাকে \highlight{Assertive sentence} বলে ।

    Assertive Sentence দুই প্রকার :

    \begin{enumerate}
        \item \highlight{Affirmative Sentence} - (হাঁ-বাচক) sentence 
        \ex{
            He gave me a pen.\\
            The earth is round.\\
            Man is mortal.
        }
        \item \highlight{Negative Sentence} (না -বাচক )Sentence 
        \ex{
            He does not take tea.\\
            Ram coul not do the sum.\\
            They are not happy.
        }
    \end{enumerate}

    \imagebox{dadu-note.png}{Assertive Sentence এর গঠন :-}{
        Subject + finite verb + object
    }
\end{bengali}

\subsection{Interogative Sentence}

\begin{bengali}
    যে sentence দ্বারা কোন প্রশ্ন জিজ্ঞাসা করা বুঝায় তাহাকে \highlight{Interogative Sentence} বলে। 

    \note{📝}{গঠন:-}{
        Interogative pronoun বা Adverb $+$ Auxiliary verb $+$ Subject $+$ Finite verb $+$ Object $+$ Interogative sign(?).
     }

     \hypertarget{topic:interopron}{\term{Interogative Pronoun}}:- Who, which, what. 

     \RightBracketBlock{
        \text{Who did the work?}\\
        \text{Which pen do you want?}\\
        \text{What do you want?}
     }{
        \text{\highlight{Interogative pronoun} \textbengali{সহযোগে গঠিত।}}
     }

     \hypertarget{topic:Interoadverb}{\term{Interogative Adverb}}:- Where, when, why, how.

     \RightBracketBlock{
        \text{Where do you go?}\\
        \text{When will you come?}\\
        \text{Why did you do the work?}\\
        \text{How will you solve the problem?}
     }{
        \text{\highlight{Interogative Adverb} \textbengali{সহযোগে গঠিত।}}
     }



     \imagebox{dadu-important.png}{Interogative pronoun "who"}
     {
        দিয়ে যে সমস্ত প্রশ্নবোধক বাক্য রচিত হয় , সে সব ক্ষেত্রে present indifinite এবং past indifinite tense-এ কোন সহাজ্যকারী ক্রিয়া (auxiliary verb) লাগে না। কিন্তু এই দুইটি tense ছাড়া বাকি সমস্ত tense-এ, tense-এর নিজস্ব অর্থ এবং চিহ্ন বজায় রাখার জন্য auxiliary verb লাগে।   
    }


    \vspace{3pt}

    
    \RightBracketBlock{
        \text{Who goes to market?}\\
        \text{Who eats rice?}\\
        \text{Who play in the field?}\\
        \text{Who wants the book?}
    }{
        \text{present indifinite tense}
    }


\RightBracketBlock{
    \text{Who told you the story?}\\
    \text{Who took coffee?}\\
    \text{Who invented television?}\\
    \text{Who got the first prize?}\\
    \text{Who stood first in the class?}
}{
    \text{past indifinite tense}
}

\end{bengali}


\begin{bengali}
    \begin{enumerate}
        \item Who \highlight{is} singing a song?
        
        এখানে Continuous tense গঠন করার জন্য auxiliary verb \highlight{'is'} এসেছে । 

        \item Who \highlight{will} help the boy?
        
        এখানে future tense গঠন করার জন্য auxiliary verb '\highlight{will}' এসেছে। 

        \item Who \highlight{has} done the sum?
        
        এখানে present prfect tense গঠনের জন্য auxiliary verb '\highlight{has}' এসেছে ।
        
        \item Who \highlight{will be} waiting for you till 5 p.m. at the station?
        
        এখানে future continuous tense গঠন করার জন্য auxiliary verb '\highlight{will be}' এসেছে । 

        \item Who \highlight{had} done the work before Ram came ?
        
        এখানে past parfect tense গঠন করার জন্য auxiliary verb '\highlight{had}' এসেছে। 

        \item Who \highlight{has been} doing the work for long three hours ?
        এখানে present parfect continuous tense গঠন করার জন্য auxiliary verb '\highlight{has been}' এসেছে। 
        
        \item Who \highlight{will have} done the work before the sun rises?
        
        এখানে future parfect tense গঠন করার জন্য auxiliary verb '\highlight{will have}' এসেছে । 
    \end{enumerate}
\end{bengali}

\pagebreak 

\subtopic{'Be' verb}
\begin{bengali}
    'Be' verb:- am, is, are, was,  were, be- এই ছয়টি হল 'Be' verb. 

    'Be' verb গুলির present, past ও past participle এর রূপ নিম্নে দেখানো হইল :-

    \begin{table}[h]
\centering
\setlength{\tabcolsep}{12pt}
\renewcommand{\arraystretch}{1.4}
\arrayrulecolor{tableline}

\begin{tabular}{L{0.3\linewidth} L{0.3\linewidth} L{0.3\linewidth}}
  \rowcolor{headerbg}
  {\color{headertext}\bfseries Present} &
  {\color{headertext}\bfseries Past} &
  {\color{headertext}\bfseries Past Participle} \\
  \midrule

  \rowcolor{subjectcol}
  Am & Was & Been \\

  \rowcolor{predicatecol}
  Is & Was & Been \\

  \rowcolor{subjectcol}
  Are & Were & Been \\

  \rowcolor{predicatecol}
  Be & Was & Been \\
  
  \bottomrule
\end{tabular}

\caption{Forms of the Verb “Be”}
\end{table}

যে সকল প্রশ্ন বোধক বাক্যের মধ্যে 'হয়' অথবা 'হওয়া' অর্থ থাকে, সে সব বাক্যে কোন \highlight{সাহায্যকারী ক্রিয়া(auxiliary verb)} লাগে না। 'হয়' বা 'হওয়া' অর্থ থাকার জন্য এই সব বাক্যে 'Be' verb টি finite রূপে গণ্য হয়। 

\imagebox{atrajit-think.png}{Note that:}{
    \term{'Be'} verb is cosidered to be an auxiliary verb as well as main verb.
}
\RightBracketBlock{
    \text{Are you weark?}\\
    \text{Is the book on the table ?}\\
    \text{Was he absent in the meeting?}\\
    \text{Were they happy?} 
}{
    \text{Auxiliary verb দ্বারা গঠিত বাক্য। }
}
\end{bengali}

\topicend

\begin{bengali}
    \note {🍕}{প্রশ্ন বোধক বাক্য গঠনকারী সাহায্যকারী ক্রিয়ার (auxiliary verb) তালিকা :-}{
        Do, did, can, could, shall, should, will, would, may, might, must, have, has, had, am, is, are, was, were, be.
    }
\end{bengali}

\topicend

\begin{bengali}
    কিছু প্রশ্নবোধক বাক্য কাছে, যেখানে কোন \hyperlink{topic:interopron}{\term{Interogative pronoun}} বা \hyperlink{topic:Interoadverb}{\term{Interogative adverb}} লাগে না। 

    \ex{
         Will  you take tea now ? \\
         Have you seen the Tajmahal?\\
         Do you like to play cricket?
    }

\end{bengali}

\topicend


\begin{bengali}
    যে সমস্ত প্রশ্নবোধক বাক্যের উত্তর 'হ্যাঁ' অথবা 'না' হয়, সে সব ক্ষেত্রে \highlight{what} লাগে না। 

    \ex{
        \\
        তুমি কি বাড়ি যাবে?\\
        Will you go home?\\
        Have you ever gone to Bombay?\\
        Did Ram do the work?

    }

    যে সমস্ত প্রশ্নবোধক বাক্যের উত্তর 'হ্যাঁ' অথবা 'না' হয় না, সে সব ক্ষেত্রে \highlight{what} লাগে। 

    \ex{
        What do you want to me?\\
        What did he eat at night?\\
        What are you doing here?
    }

\end{bengali}

\topicend

\subtopic{Possessive Pronouns/Possessive Adjectives}
% personal pronoun different cases table
\begin{table}[h]
\centering
\setlength{\tabcolsep}{8pt}
\renewcommand{\arraystretch}{1.6}
\arrayrulecolor{tableline}

\begin{minipage}[c]{0.08\textwidth}
\rotatebox{90}{\parbox{4.5cm}{\centering\small\color{primary} This note does not refer to sentence}}
\end{minipage}%
\begin{minipage}[c]{0.88\textwidth}
\begin{tabular}{
  L{0.20\linewidth}|
  L{0.22\linewidth}|
  L{0.22\linewidth}|
  L{0.18\linewidth}|
}
  \hline
  \rowcolor{headerbg}
  \multicolumn{1}{|c|}{\color{headertext}\bfseries Nominative} &
  \multicolumn{1}{c|}{\color{headertext}\bfseries Possessive} &
  \multicolumn{1}{c|}{\color{headertext}\bfseries Possessive} &
  \multicolumn{1}{c|}{\color{headertext}\bfseries Objective} \\
  
  \rowcolor{headerbg}
  \multicolumn{1}{|c|}{\color{headertext}\bfseries Case} &
  \multicolumn{1}{c|}{\color{headertext}\bfseries Adjective} &
  \multicolumn{1}{c|}{\color{headertext}\bfseries Pronoun} &
  \multicolumn{1}{c|}{\color{headertext}\bfseries Case} \\
  \hline

  \rowcolor{subjectcol}
  I & my & mine & me \\

  \rowcolor{predicatecol}
  We   & our   & ours   & us \\

  \rowcolor{subjectcol}
  You  & your  & yours  & you \\

  \rowcolor{predicatecol}
  He   & his   & his    & him \\

  \rowcolor{subjectcol}
  She  & her   & hers   & her \\

  \rowcolor{predicatecol}
  They & their & theirs & them \\
  \hline
\end{tabular}
\end{minipage}

\caption{Personal Pronouns in Different Cases}
\end{table}


\begin{bengali}
    My, our, your, his, their, her- এই pronoun গুলিকে বলা হয় \highlight{possessive adjective}. ইহারা একসঙ্গে adjective এর মতো কোন noun বা pronoun কে qualify করে (দোষ গুণ প্রকাশ করে ), ষষ্ঠী বিভক্তির কাজও করে। কাজেই ইহাদের \highlight{double parts of speech} বলা হয়। 

    \ex{
        My pen, our country, his dog, your book, their table clock, her frock.
    }

    উপরি উক্ত possesive adjective গুলার পরে সব সময় একটি noun অথবা pronoun বসে। 


    Mine, ours, yours, theirs, hers/his- ইহাদের বলা হয় \highlight{possesive pronoun}। ইহাদের পরে কোন noun বা pronoun বসে না। 

    \ex{
        It is a pen of mine. It is a country of ours. Yours faithfully. etc.
    }
\end{bengali}


\imagebox{atrajit-think.png}{Double Parts of speech}
{
    The possesive pronouns work in a sentence as an adjective-fused noun. Meaning, \highlight{Mine is better than yours}- in this sentence note that the word \term{Mine} and \term{yours} are working as nouns but at the same time they are qualifying the noun they are indicating at. For example, \highlight{Mine} indicate my book/ my pen/ my house/ my will of fire etc. Same goes for \highlight{yours}. So they are indicating the objects like book/pen/house/will of fire at the same time they are also qualifing the object to indicate whose object it is. Ror example: Mine indicates it is my pen/book/house/will of fire etc. Mine $=$ My(possesive adjective) $+$ noun (object e.g. pen/book/will of fire etc.)
}



\subsection{Imprerative Sentence}
\term{Definition}:-
\begin{bengali}
    যে বাক্যের দ্বারা আদেশ, উপদেশ এবং অনুরোধ বোঝায় তাহাকে \highlight{Imperative Sentence} বলে। 

    \term{গঠন}:-

    Subject 'you' ঊহ্য় $+$ finite verb $+$ object. 

    Imperative sentence এর verb টির সর্বদা present finite tense হয়। 
\end{bengali}

\ex{
    Come here.\\
    Do it now.\\
    Do not tell a lie.\\
    Please help me.
}

\begin{bengali}
    \imagebox{dadu-note.png}{'Let' দ্বারাও Imperative sentence গঠিত হয়। যথা:- }{
        Let me go.\\
        Let us play.\\
        Let them go.\\
        Let us take tea.
    }

    \vspace{15pt}

    \imagebox{atrajit-think.png}{Use of "Let"}{
        When we start an Imprerative sentence with 'Let' to actually indicate a request to give someone permission to do some action. But when we use Let followed by us, we indicate a suggestion not request. Why such chnage? Because, in that case (unless specify) we request to ourselves to give permission to ourselves to go some action. This turns out to be a way of suggestion actually. But we write \highlight{Mom, let us go.}Then definitely we are not requesting ourselves instead we are requesting to mom. 

        \note{⚠️}{Warning:}{
            If we write like this: \highlight{Mom! Let us go.} This means I am suggesting to mom to go with me.
        }
    }
\end{bengali}


\subsection{Optative Sentence}
\term{Definition}:-
\begin{bengali}
    যে sentence  দ্বারা মনের ইচ্ছা এবং প্রার্থনা বুঝায় তাহাকে \highlight{Optative sentence} বলে। 

    \term{গঠন}:-
    \ex{
        \begin{enumerate}
            \item May you live long.
            \item \hypertarget{may:exception}{Long live our king.}
            \item May he not be punished.
            \item May you shine in life.
        \end{enumerate}
    }

    \note{📌}{Note:-}{Optative sentence-এ অনেক সময় 'May' ঊহ্য থাকে। Refer to \hyperlink{may:exception}{example no-2}}
\end{bengali}


\subsection{Exclamatory Sentence}
\term{Definition}:-
\begin{bengali}
    যে বাক্যের দ্বারা (sentence) মনের হর্ষ, বিষাদ, ভয়, ঘৃণা, আবেগ এবং বিষ্ময় প্রকাশ পায় তাহাকে \highlight{Exclamatory sentence} বলে। 

    \term {গঠন}:-

    \begin{enumerate}
        \item What বা how $+$ বাক্যে প্রদত্ত adjective $+$ subject $+$ finite verb $+$ object  $+$ বিষ্ময় চিহ্ন ('!')। 
        \ex{
        \\
        What a beautiful scenery it is!\\
        How brave the boy was!\\
        What an excellent idea it is!\\
        How funny the story is!\\
        What a nice bird it is!
        }
        \item আর এক প্রকার বিস্ময় বোধক বাক্য (exclamatory sentence), বিস্ময় বোধক শব্দ সহযোগে গঠিত হয়। 
        
        বিস্ময় বোধক শব্দের তালিকা (List of exclamatory term):

        $$
        \begin{array}{llcll}
            \text{Alas}   && - && \text{হায় হায়}, \\
            \text{Hurrah} && - && \text{কি আনন্দ }, \\
            \text{Hush}   && - && \text{চুপ},\\
            \text{Fie} && - && \text{ছিঃ},\\
            \text{Bravo} && - && \text{সাবাশ},\\
            \text{Oh}&& -&&\text{আহা ।}
        \end{array}
        $$

    \end{enumerate}

    \term{গঠন}:-

    Exclamatory term $+$ Assertive.

    Alas! I am undone!\\
    Hurrah! We have won the game!\\
    Hush! \hypertarget{mynickname}{Papai} is sleeping!\\
    Fie! You are a liar!\\
    Bravo! You have done well!\\
    Oh! What charming scenery!
\end{bengali}

\imagebox{atrajit-smile.png}{Thank you dadu!}{
    \hyperlink{mynickname}{Papai} is my nick name. I am so happy that you mentioned my name in your precious notes as an example thus making my name forever memorable to the readers.
}

\topicend

\subsection{Imphatic Sentence}
\term{Definition}:-
\begin{bengali}
    যে sentence  - এ উহার অন্তর্গত word গুলির স্বাভাবিক অবস্থিতির পরিবর্তন ঘটাইয়া বা নূতন কোন exphasing word যোগ করিয়া sentence - এ জোর দেওয়া হয় তাহাকে emphatic sentence বলে। \cite{AKG}
    
    \ex{
        \\
        Please do come in, \hypertarget{myname}{Atrajit Sarkar}.\\
        Do take my love.\\
        Go you must.\\
        Please do take your seat.
    }
\end{bengali}

\imagebox{atrajit-smile.png}{Thank you again dadu!}{
    Once again you used my name \hyperlink{myname}{Atrajit Sarkar}. This is my real name. So, it will also be memorable to readers. Hurrah!
}




\topicend

